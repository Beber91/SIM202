\documentclass{article}
\usepackage[utf8]{inputenc}
\usepackage[french]{babel}
\usepackage[T1]{fontenc}
\usepackage{amsmath}
\usepackage{amsfonts}
\usepackage{amssymb}
\usepackage{amsthm}
\usepackage{mathrsfs}
\usepackage{mathabx}
\usepackage{stackengine}
\usepackage{dsfont}
\usepackage{scalerel}
\usepackage{tikz}
\usetikzlibrary{calc}
\usepackage{stmaryrd}
\usepackage{hyperref}
\usepackage{mathrsfs}
\usepackage{graphicx}
\usepackage{xcolor}
\usepackage{wrapfig}
\usepackage{wasysym}
\usepackage{enumitem}
\usepackage[a4paper]{geometry}
\usepackage{float}
\frenchbsetup{StandardLists=true}
\geometry{hmargin=2.5cm,vmargin=2.3cm}
\setlength{\parskip}{0cm}
\newcommand{\abs}[1]{\vert#1\vert}
\newcommand{\norme}[1]{\left\Vert#1\right\Vert}
\newcommand{\scal}[2]{\left<#1\vert#2\right>}
\newcommand{\overbow}[1]{
	\tikz [baseline = (N.base), every node/.style={}] {
		\node [inner sep = 0pt] (N) {$#1$};
		\draw [line width = 0.4pt] plot [smooth, tension=1.3] coordinates {
			($(N.north west) + (0.1ex,0)$)
			($(N.north)      + (0,0.5ex)$)
			($(N.north east) + (0,0)$)
		};
	}
}
\newcommand{\normee}[1]{\left\vvvert #1 \right\vvvert}
\newcommand\dangersign[1][2ex]{%
	\renewcommand\stacktype{L}%
	\scaleto{\stackon[1.3pt]{\color{red}$\triangle$}{\tiny !}}{#1}%
}
\newcommand{\transp}[1]{{}^t\,\!#1}
\newcommand{\intt}{\displaystyle\int}
\AtBeginDocument{%
	\LetLtxMacro\oldref{\ref}%
	\DeclareRobustCommand{\ref}[2][]{(\oldref#1{#2})}%
}
\usepackage{Sweave}
\begin{document}
\Sconcordance{concordance:rapport.tex:rapport.Rnw:%
1 52 1 1 0 41 1 1 6 8 0 1 3 366 0 1 2 3 0 1 12 1 3 2 1 1 2 38 0 1 2 4 1 %
1 2 4 1 1 4 1 1 1 4 1 7 3 1 1 6 2 1}

\hypersetup{pageanchor=false}
\begin{titlepage}
	\newcommand{\HRule}{\rule{\linewidth}{0.5mm}}
	\centering
	\textsc{\LARGE Université Paris Saclay}\\[1.5cm]
	\includegraphics[width=200px]{img/LogoMathOrsay.png}
	\HRule \\[0.4cm]
	{ \huge \bfseries\centering Lockdown Period Load Forecasting}\\[0.4cm]
	\HRule \\[1.5cm]
	\begin{minipage}{0.4\textwidth}
		\begin{flushleft} \large
			\emph{Auteur :}\\
			Béreux \textsc{Nicolas}
			\\
			Denis \textsc{Matthieu}
		\end{flushleft}
	\end{minipage}
	\begin{minipage}{0.4\textwidth}
		\begin{flushright} \large
			\emph{Superviseur:} \\
			\href{https://www.imo.universite-paris-saclay.fr/~goude/about.html}{Yannig \textsc{Goude}} \\
		\end{flushright}
	\end{minipage}\\[2cm]
	{\large \today}\\[2cm]
	\includegraphics[width=200px]{img/logo_Paris_saclay.png}
\end{titlepage}
\hypersetup{pageanchor=true}
\tableofcontents

\newtheorem{theorem}{Théorème}[section]
\newtheorem{corollary}[theorem]{Corollaire}
\newtheorem{lemma}[theorem]{Lemme}
\newtheorem{prop}[theorem]{Proposition}
\newtheorem{example}[theorem]{Exemple}
\theoremstyle{definition}
\newtheorem{definition}{Définition}
\theoremstyle{remark}
\newtheorem*{remark}{Remarque}
\pagebreak
\section{Présentation du sujet}
\begin{Schunk}
\begin{Soutput}
   tidyverse    lubridate       ranger       pracma      Metrics         mgcv 
        TRUE         TRUE         TRUE         TRUE         TRUE         TRUE 
       keras       visreg        caret         mc2d        opera        abind 
        TRUE         TRUE         TRUE         TRUE         TRUE         TRUE 
randomForest   tensorflow       plot3D        e1071          xts 
        TRUE         TRUE         TRUE         TRUE         TRUE 
\end{Soutput}
\begin{Soutput}
$cross_validation.r
$cross_validation.r$value
function (data) 
{
    set.seed(123)
    to.train = rbern(n = length(data$Load), p = 0.8) == T
    cross.train = data[to.train, ]
    cross.test = data[!to.train, ]
    return(list(train = cross.train, test = cross.test))
}

$cross_validation.r$visible
[1] FALSE


$days_to_numeric.r
$days_to_numeric.r$value
function (data) 
{
    data$WeekDays[data$WeekDays == "Monday"] = 1
    data$WeekDays[data$WeekDays == "Tuesday"] = 2
    data$WeekDays[data$WeekDays == "Wednesday"] = 3
    data$WeekDays[data$WeekDays == "Thursday"] = 4
    data$WeekDays[data$WeekDays == "Friday"] = 5
    data$WeekDays[data$WeekDays == "Saturday"] = 6
    data$WeekDays[data$WeekDays == "Sunday"] = 7
    data$WeekDays = as.numeric(data$WeekDays)
    return(data$WeekDays)
}

$days_to_numeric.r$visible
[1] FALSE


$descriptive_analysis.r
$descriptive_analysis.r$value
function (ts) 
{
    month = as.factor(.indexmon(ts))
    mean.month = tapply(ts, month, mean)
    noise = c()
    for (i in c(1:length(ts))) {
        noise = c(noise, mean.month[month[i]] - ts[i])
    }
    return(noise)
}

$descriptive_analysis.r$visible
[1] FALSE


$evaluate.r
$evaluate.r$value
function (test_label, predicted_set) 
{
    return(rmse(test_label, predicted_set))
}

$evaluate.r$visible
[1] FALSE


$format_data.r
$format_data.r$value
function (file_train, file_test) 
{
    print(paste("Load and format ", file_train, sep = " "))
    train_set = read_csv(file_train, col_types = cols())
    train_set$WeekDays[train_set$WeekDays == "Monday"] <- 1
    train_set$WeekDays[train_set$WeekDays == "Tuesday"] <- 2
    train_set$WeekDays[train_set$WeekDays == "Wednesday"] <- 3
    train_set$WeekDays[train_set$WeekDays == "Thursday"] <- 4
    train_set$WeekDays[train_set$WeekDays == "Friday"] <- 5
    train_set$WeekDays[train_set$WeekDays == "Saturday"] <- 6
    train_set$WeekDays[train_set$WeekDays == "Sunday"] <- 7
    train_set$WeekDays = as.integer(train_set$WeekDays)
    train_set$Year = NULL
    train_set$Date = NULL
    train_label = data.matrix(train_set$Load)
    train_set$Load = NULL
    train_set = data.matrix(train_set)
    print(paste("Load and format ", file_test, sep = " "))
    test_set = read_csv(file_test, col_types = cols())
    test_set$WeekDays[test_set$WeekDays == "Monday"] <- 1
    test_set$WeekDays[test_set$WeekDays == "Tuesday"] <- 2
    test_set$WeekDays[test_set$WeekDays == "Wednesday"] <- 3
    test_set$WeekDays[test_set$WeekDays == "Thursday"] <- 4
    test_set$WeekDays[test_set$WeekDays == "Friday"] <- 5
    test_set$WeekDays[test_set$WeekDays == "Saturday"] <- 6
    test_set$WeekDays[test_set$WeekDays == "Sunday"] <- 7
    test_set$WeekDays = as.integer(test_set$WeekDays)
    test_label = test_set$Load.1
    tmp = test_set$Load.1
    for (i in c(1:(length(tmp) - 1))) {
        test_label[i] = tmp[i + 1]
    }
    test_set$Year = NULL
    test_set$Date = NULL
    test_set$Id = NULL
    test_set$Usage = NULL
    test_set = data.matrix(test_set)
    test_set = data.matrix(test_set)
    return(list(train_set = train_set, train_label = train_label, 
        test_set = test_set, test_label = test_label))
}

$format_data.r$visible
[1] FALSE


$fourier.r
$fourier.r$value
function (train, test, plt = FALSE) 
{
    total.time = c(1:(nrow(train) + nrow(test)))
    length(total.time)
    train$time = total.time[1:nrow(train)]
    test$time = tail(total.time, nrow(test))
    fourier.make.matrix = function(t, k, period) {
        w = 2 * pi/period
        ret = cbind(cos(w * t), sin(w * t))
        for (i in c(2:K)) {
            ret = cbind(ret, cos(i * w * t), sin(i * w * t))
        }
        return(ret)
    }
    K = 5
    period = 365
    fourier.train = fourier.make.matrix(train$time, K, period)
    fourier.test = fourier.make.matrix(test$time, K, period)
    fourier.train.df = data.frame(train$Load, fourier.train)
    fourier.test.df = data.frame(fourier.test)
    reg = lm(train.Load ~ ., data = fourier.train.df)
    pred.fourier = predict(reg, newdata = fourier.test.df)
    total.fourier = c(reg$fitted, pred.fourier)
    if (plt) {
        par(mfrow = c(1, 1))
        plot(train$Load, type = "l", xlim = c(0, length(total.time)))
        lines(reg$fitted, col = "red", lwd = 2)
        lines(test$time, pred.fourier, col = "green", lwd = 2)
    }
    return(pred.fourier)
}

$fourier.r$visible
[1] FALSE


$GAM.r
$GAM.r$value
function (train, test, plt = FALSE) 
{
    Gam <- gam(Load ~ s(Load.1) + s(Load.7) + s(Temp) + s(Temp_s95) + 
        s(WeekDays, k = 7) + s(GovernmentResponseIndex), data = train)
    summary(Gam)
    gam.train = predict(Gam, newdata = train)
    gam.test = predict(Gam, newdata = test)
    if (plt) {
        par(mfrow = c(1, 1))
        plot(train$Load, type = "l", xlim = c(0, length(total.time)))
        lines(train$time, Gam$fit, col = "red", lwd = 1)
        lines(test$time, gam.test, col = "green", lwd = 1)
    }
    return(gam.test)
}

$GAM.r$visible
[1] FALSE


$lstm.r
$lstm.r$value
function (train, test, plt = FALSE) 
{
    labels = train[, 2]
    train.lstm = train[, -c(1, 2, 22, 23)]
    test.lstm = test[, -c(1, 20, 21, 23, 24)]
    lstm = function(train_set, train_label, test_set) {
        window = 10
        n_windows = nrow(train_set) - window + 1
        n_features = ncol(train_set)
        x = array(data = NA, dim = c(n_windows, window, n_features))
        y = array(data = NA, dim = c(n_windows, window, 1))
        for (i in 1:n_windows) {
            x[i, , ] = data.matrix(train_set[i:(i + window - 
                1), ])
            y[i, , ] = as.matrix(data.matrix(train_label[i:(i + 
                window - 1), ]))
        }
        i = 1
        print(size(y))
        batch_size = 1
        lrelu = function(x) tf.keras.activations.relu(x, alpha = 0.1)
        model = keras_model_sequential() %>% layer_lstm(units = 64, 
            batch_input_shape = c(batch_size, window, n_features), 
            dropout = 0.2, recurrent_dropout = 0.2, return_sequences = T, 
            ) %>% time_distributed(layer_dense(units = 1))
        model %>% compile(loss = "mae", optimizer = optimizer_rmsprop())
        model %>% fit(x, y, epochs = 15)
        pred.train = model %>% predict(X_train)
        pred.test = model %>% predict(X_test)
        return(list(train = pred.train, test = pred.test, model = model))
    }
    pred.lstm = lstm(train.lstm, labels, test.lstm)
    if (plt) {
        par(mfrow = c(1, 1))
        plot(train$Load, type = "l", xlim = c(0, length(total.time)))
        lines(test$time, pred.lstm$test, col = "green", lwd = 1)
    }
    N = length(test$Load.1)
    RMSE = rmse(pred.lstm$test[-N], test$Load.1[2:N])
    RMSE
    return(pred.lstm)
}

$lstm.r$visible
[1] FALSE


$main.r
NULL

$neural_network.r
$neural_network.r$value
function (train, test, plt = FALSE) 
{
    labels = select(train, "Load")
    train.lstm = train
    test.lstm = test
    lstm = function(train_set, train_label, test_set) {
        window = nrow(test_set)
        n_windows = nrow(train_set) - window + 1
        n_features = ncol(train_set)
        x_train = array(data = NA, dim = c(n_windows, window, 
            n_features))
        y_train = array(data = NA, dim = c(n_windows, window, 
            1))
        for (i in window:nrow(train_set)) {
            x_train[i - window + 1, , ] = data.matrix(train_set[(i - 
                window + 1):i, ])
            y_train[i - window + 1, , ] = as.matrix(data.matrix(train_label[(i - 
                window + 1):i, ]))
        }
        x_test = array(data = NA, dim = c(1, window, n_features))
        for (i in window:nrow(test_set)) {
            x_test[i - window + 1, , ] = data.matrix(test_set[(i - 
                window + 1):i, ])
        }
        batch_size = 40
        n = size(x_train)[1]
        to_remove = sample.int(n, n%%batch_size)
        x_train = x_train[-to_remove, , ]
        y_train = y_train[-to_remove, , ]
        y_train = array(y_train, dim = c(size(y_train)[1], size(y_train)[2], 
            1))
        model = keras_model_sequential() %>% layer_lstm(units = 64, 
            batch_input_shape = c(batch_size, window, n_features), 
            return_sequences = T, activation = layer_activation_relu(max_value = 40000), 
            kernel_regularizer = regularizer_l2(0.001), bias_regularizer = regularizer_l2(0.001)) %>% 
            layer_lstm(units = 64, return_sequences = T, activation = layer_activation_relu(max_value = 40000), 
                kernel_regularizer = regularizer_l2(0.001), bias_regularizer = regularizer_l2(0.001)) %>% 
            layer_lstm(units = n_features, return_sequences = T, 
                activation = layer_activation_relu(max_value = 40000), 
                kernel_regularizer = regularizer_l2(0.001), bias_regularizer = regularizer_l2(0.001)) %>% 
            time_distributed(layer_dense(units = 1))
        model %>% compile(loss = "mse", optimizer_rmsprop(lr = 0.01, 
            rho = 0.9, epsilon = NULL, decay = 0, clipnorm = 1, 
            clipvalue = NULL))
        model %>% fit(x_train, y_train, epochs = 4, batch_size = batch_size)
        print("www")
        model_evaluate = keras_model_sequential() %>% layer_lstm(units = 64, 
            batch_input_shape = c(1, window, n_features), return_sequences = T, 
            activation = layer_activation_relu(max_value = 40000), 
            kernel_regularizer = regularizer_l2(0.001), bias_regularizer = regularizer_l2(0.001)) %>% 
            layer_lstm(units = 64, return_sequences = T, activation = layer_activation_relu(max_value = 40000), 
                kernel_regularizer = regularizer_l2(0.001), bias_regularizer = regularizer_l2(0.001)) %>% 
            layer_lstm(units = n_features, return_sequences = T, 
                activation = layer_activation_relu(max_value = 40000), 
                kernel_regularizer = regularizer_l2(0.001), bias_regularizer = regularizer_l2(0.001)) %>% 
            time_distributed(layer_dense(units = 1))
        print("www1")
        set_weights(model_evaluate, get_weights(model))
        print("www2")
        pred.test = model_evaluate %>% predict(x_test, batch_size = 1) %>% 
            .[, , 1]
        print("www4")
        return(list(train = 1, test = pred.test, model = model_evaluate))
    }
    pred.lstm = lstm(train.lstm, labels, test.lstm)
    if (plt) {
        par(mfrow = c(1, 1))
        plot(train$Load, type = "l", xlim = c(0, length(total.time)))
        lines(test$time, pred.lstm$test, col = "green", lwd = 1)
    }
    N = length(test$Load.1)
    RMSE = rmse(pred.lstm$test[-N], test$Load.1[2:N])
    RMSE
    return(pred.lstm)
}

$neural_network.r$visible
[1] FALSE


$random_forest.r
$random_forest.r$value
function (train, test, n_trees, plt = FALSE) 
{
    rf = randomForest(Load ~ ., data = train, mtry = n_trees, 
        importance = TRUE, na.action = na.omit)
    pred.test.rf = predict(rf, test)
    print(rf$importance)
    if (plt) {
        par(mfrow = c(1, 1))
        plot(train$Load, type = "l", xlim = c(0, length(total.time)))
        lines(test$time, pred.test.rf, col = "green", lwd = 1)
    }
    N = length(test$Load.1)
    RMSE = rmse(pred.test.rf[-N], test$Load.1[2:N])
    RMSE
    return(list(pred.test.rf = pred.test.rf, rf = rf))
}

$random_forest.r$visible
[1] FALSE


$scale.r
$scale.r$value
function (scaled, scaler, feature_range = c(0, 1)) 
{
    min = scaler[1]
    max = scaler[2]
    t = length(scaled)
    mins = feature_range[1]
    maxs = feature_range[2]
    inverted_dfs = numeric(t)
    for (i in 1:t) {
        X = (scaled[i] - mins)/(maxs - mins)
        rawValues = X * (max - min) + min
        inverted_dfs[i] <- rawValues
    }
    return(inverted_dfs)
}

$scale.r$visible
[1] FALSE


$xgboost.r
$xgboost.r$value
function (train_set, train_label, test_set) 
{
    param = list(booster = "gblinear", objective = "reg:squarederror", 
        eval_metric = "rmse", lambda = 3e-04, alpha = 3e-04, 
        nthread = 2, eta = 0.1)
    print("Model : XGBOOST")
    xgbmodel = xgboost(data = train_set, label = train_label, 
        nrounds = 200, params = param, verbose = 0)
    pred = predict(xgbmodel, test_set)
    return(pred)
}

$xgboost.r$visible
[1] FALSE
\end{Soutput}
\begin{Soutput}
[1] "Load and format  ./data/train_V2.csv"
[1] "Load and format  ./data/test_V2.csv"
\end{Soutput}
\end{Schunk}
Nous allons nous intéresser à la prédiction de la consommation électrique en France durant la pandémie de COVID-19. Cette situation de confinement et couvre-feu étant inédite, nous n'avons pas de données et de modèles adaptés. Nous allons donc essayer de fournir un modèle prédictif dont nous évaluerons la qualité en utilisant la Root Mean Squared Error (RMSE). Prédire la consommation électrique à l'avance permet d'adapter la production en amont et ainsi d'éviter des coupures de courant en cas de demande élevée et d'éviter d'en produire trop. Les centrales à énergie fossile (gaz, charbon) étant les plus rapides à mettre en marche et arrêter, on peut ainsi réduire au minimum l'émission de gaz à effet de serre.
\par
Ce jeu de données comporte 3028 ligne pour la partie d'entrainement et 275 lignes pour la partie de test.  Chaque ligne est composée de 
\begin{Schunk}
\begin{Soutput}
     Index                         Load.1          Load.7           Temp       
 Min.   :2012-01-01 00:00:00   Min.   :35589   Min.   :35589   Min.   :-4.897  
 1st Qu.:2014-01-26 18:00:00   1st Qu.:46727   1st Qu.:46757   1st Qu.: 7.830  
 Median :2016-02-22 12:00:00   Median :51261   Median :51319   Median :12.084  
 Mean   :2016-02-22 12:00:00   Mean   :54617   Mean   :54669   Mean   :12.542  
 3rd Qu.:2018-03-20 06:00:00   3rd Qu.:63140   3rd Qu.:63173   3rd Qu.:17.498  
 Max.   :2020-04-15 01:00:00   Max.   :94097   Max.   :94097   Max.   :28.066  
    Temp_s95         Temp_s99       Temp_s95_min     Temp_s95_max   
 Min.   :-4.522   Min.   :-4.152   Min.   :-6.186   Min.   :-3.782  
 1st Qu.: 7.824   1st Qu.: 7.890   1st Qu.: 6.704   1st Qu.: 8.933  
 Median :12.076   Median :12.035   Median :10.717   Median :13.460  
 Mean   :12.542   Mean   :12.541   Mean   :11.155   Mean   :13.935  
 3rd Qu.:17.519   3rd Qu.:17.596   3rd Qu.:15.886   3rd Qu.:19.064  
 Max.   :27.985   Max.   :26.318   Max.   :25.438   Max.   :30.514  
  Temp_s99_min     Temp_s99_max         toy              WeekDays    
 Min.   :-4.518   Min.   :-3.732   Min.   :0.001338   Min.   :1.000  
 1st Qu.: 7.615   1st Qu.: 8.253   1st Qu.:0.230859   1st Qu.:2.000  
 Median :11.652   Median :12.504   Median :0.481530   Median :4.000  
 Mean   :12.175   Mean   :12.978   Mean   :0.487565   Mean   :3.999  
 3rd Qu.:17.190   3rd Qu.:18.048   3rd Qu.:0.741110   3rd Qu.:6.000  
 Max.   :25.630   Max.   :27.087   Max.   :0.998662   Max.   :7.000  
       BH              Month             DLS        Summer_break    
 Min.   :0.00000   Min.   : 1.000   Min.   :1.00   Min.   : 0.0000  
 1st Qu.:0.00000   1st Qu.: 3.000   1st Qu.:1.00   1st Qu.: 0.0000  
 Median :0.00000   Median : 6.000   Median :2.00   Median : 0.0000  
 Mean   :0.03534   Mean   : 6.375   Mean   :1.57   Mean   : 0.9247  
 3rd Qu.:0.00000   3rd Qu.: 9.000   3rd Qu.:2.00   3rd Qu.: 0.0000  
 Max.   :1.00000   Max.   :12.000   Max.   :2.00   Max.   :10.0000  
 Christmas_break   GovernmentResponseIndex
 Min.   : 0.0000   Min.   : 0.000         
 1st Qu.: 0.0000   1st Qu.: 0.000         
 Median : 0.0000   Median : 0.000         
 Mean   : 0.9181   Mean   : 1.103         
 3rd Qu.: 0.0000   3rd Qu.: 0.000         
 Max.   :20.0000   Max.   :72.500         
\end{Soutput}
\end{Schunk}
Les données d'entrainement vont du 1er janvier 2012 au 15 avril 2020 (soit 1 mois après le début du premier confinement). 
 Nous vérifions ensuite si il y a des valeurs invalides dans notre jeu de données: 
 <<apercu,echo=FALSE>>=
 which(is.na(train_set))
 which(is.na(test_set))
Il n'y en a aucune.

Nous allons maintenant nous concentrer uniquement sur les données d'entrainement.
\section{Analyse descriptive des données}
Commençons par regarder à quoi ressemble nos données
\subsection{Saisonnalité}
Nous observons une saisonnalité annuelle. Si nous nous concentrons sur deux mois :
Nous observons ici une saisonnalité au niveau de la semaine. Vérifions cela en traçant les consommations moyennes sur tout le jeu de données
Ces figures confirment une saisonnalité à l'année avec une consommation électrique plus élevée en hiver et chutant en été, particulièrement en août. Nous pouvons corréler cela avec la température : d'après \href{https://travaux.edf.fr/electricite/raccordement/repartition-de-la-consommation-d-electricite-au-sein-d-un-foyer-francais}{EDF}, le chauffage électrique correspond à 62$\%$ de la consommation électrique au sein d'une maison ou d'un appartement en France. Nous observons aussi une baisse de la consommation électrique lors du week-end (car moins de personnes travaillent le week-end).

\subsection{Corrélation}

\subsection{Statistiques de base}

\end{document}
